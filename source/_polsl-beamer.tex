 
%
\documentclass[aspectratio=169]{beamer}
%\documentclass[aspectratio=1610]{beamer}
%\documentclass[aspectratio=149]{beamer}
%\documentclass[aspectratio=54]{beamer}
%\documentclass[aspectratio=43]{beamer}
%\documentclass[aspectratio=32]{beamer} 


\usepackage[utf8]{inputenc}                                      
\usepackage[T1]{fontenc}  
\usepackage[polish]{babel} 


\ifxetex  % XeLaTeX
   \usepackage{fontspec}
   \defaultfontfeatures{Mapping=tex--text}   % to support TeX conventions like dashes etc
   \usepackage{xltxtra} % extra customisation for XeLaTeX
%   \setsansfont{Linux Biolinum O}
%   \setmainfont[Ligatures={Common,TeX}, Numbers={OldStyle}]{Linux Libertine O}
%      \setsansfont{Barlow Semi Condensed}
%      \setmainfont{PT Serif}
%   \setsansfont{Ubuntu}
%   \setmonofont{Ubuntu Mono}

\else  % LaTeX
   \usepackage[T1]{fontenc}
   \usepackage[utf8]{inputenc}
%   \usepackage[osf,tt=false]{libertine}
\fi


\uselanguage{polish}
\languagepath{polish}
\deftranslation[to=polish]{Definition}{Definicja}
\deftranslation[to=polish]{Example}{Przykład}

\usetheme[sectionpage=progressbar,subsectionpage=progressbar,block=fill,background=light]{metropolis}





\usepackage{indentfirst}
%\usepackage{graphicx} 
%\usepackage{hyperref}
%\usepackage{booktabs}
%\usepackage{tikz}
%\usetikzlibrary{arrows}
%\usepackage{pgfplots}
%\usepackage{subfigure}
%\usepackage{color}
%\usepackage{listings}
%\usepackage{filecontents}
%\usepackage{fontawesome5}
%\usepackage{amssymb}



%\usepackage[backend=bibtex,style=alphabetic]{biblatex}
%plain --> \usepackage[style=numeric]{biblatex}
%abbrv --> \usepackage[style=numeric,firstinits=true]{biblatex}
%unsrt --> \usepackage[style=numeric,sorting=none]{biblatex}
%alpha --> \usepackage[style=alphabetic]{biblatex}

%\newrobustcmd*
\newcommand%
{\footfullcitenomark}{%
  \AtNextCite{%
    \let\thefootnote\relax
    \let\mkbibfootnote\mkbibfootnotetext}%
  \footfullcite}
\newcommand{\stopka}[1]{\footfullcitenomark{#1}}  



\newcommand{\hcancel}[1]{%
   \tikz[baseline=(tocancel.base)]{
      \node[inner sep=0pt,outer sep=0pt] (tocancel) {#1};
      \draw[red, thick] (tocancel.south west) -- (tocancel.north east);
   }%
}%


%%%%%%%%%%%%%%%%%%%%%%%%%%%%

 

%\setbeamertemplate{footline}
%    {\begin{beamercolorbox}[sep=1ex]{author in head/foot}
%      \rlap{\textit{\insertshorttitle}}\hfill\insertauthor\hfill\llap{\insertframenumber}%
%      \end{beamercolorbox}%
%}

 
\newcommand{\HUGE}{\fontsize{35}{30}\selectfont}
%\newcommand{\veryHuge}{\fontsize{65}{55}\selectfont}
\newcommand{\veryHuge}{\fontsize{45}{40}\selectfont}

\usepackage{color}
\definecolor{cornflowerblue}{cmyk}{0.65, 0.13, 0   , 0   }

\newcommand*{\myfont}{\fontfamily{LinuxLibertineT-OsF}\selectfont}

\newcommand{\cytowanie}[1]{%
   \raisebox{-0.2cm}{\myfont\color{blue}\veryHuge «}\myfont\Large #1\raisebox{-0.2cm}{\myfont\color{blue}\veryHuge »}
} 



\title[tytuł krótki]{Tytuł długi}
\subtitle{podtytuł}
\author{autor(zy)}
\institute[PolSl]{\includegraphics[scale=0.25]{graf/politechnika_sl_logo_poziom_pl}}


%\date{23 marca 2021}
\logo{\includegraphics[scale=0.2]{graf/politechnika_sl_logo_pion_pl.eps}}

\frenchspacing

\begin{document}
\maketitle

\begin{frame}{Spis treści}
  \setbeamertemplate{section in toc}[sections numbered]
  \tableofcontents[hideallsubsections]
\end{frame}

\begin{frame}[fragile]{Proporcje slajdów}
Propocje slajdów można ustawić na samym początku kodu źródłowego:

\begin{verbatim}
\documentclass[aspectratio=169]{beamer}  % 16:9
\documentclass[aspectratio=1610]{beamer} % 16:10
\documentclass[aspectratio=149]{beamer}  % 14:9
\documentclass[aspectratio=54]{beamer}   %  5:4
\documentclass[aspectratio=43]{beamer}   %  4:3
\documentclass[aspectratio=32]{beamer}   %  3:2
\end{verbatim}
\end{frame} 



\section{Sekcja 1}

\begin{frame}{\LaTeX\ifxetex, \XeLaTeX\fi}

Szablon można skompilować zarówno \LaTeX em, jak i Xe\LaTeX em. Będą użyte inne rodziny fontów. 

Xe\LaTeX\ \ifxetex(\XeLaTeX) \fi używa fontu Fira, \LaTeX\ używa domyślnego fontu.

\ifxetex
Unicode: 
źdźbło żółć fuŝĥoraĵo æbletræ ¡¿piña?! smršť à ă ä á â ǎ ą å ā ã ạ ȧ æ
ì ï ĩ ī ĭ í į ı i ǐ î  
ò ö õ ō ó ő ø ŏ ǒ ô ǫ 
đ ð p þ b ťęľěģŕåþħ
şœșŋøŵý
äße 
γλῶττα
љубав међа 
ѕвезда їжак
\fi 
\end{frame}

\subsection{Podsekcja 1}

\begin{frame}{Tytuł}
\begin{itemize}
\item raz
\item dwa
\item trzy \alert{alert}
\end{itemize}
\end{frame}

\subsection{Podsekcja 2}



\begin{frame}{Wzory}
\begin{align*}
A = \int_{-\infty}^{0} \exp \left( - \frac{(x-m)^2}{2\sigma^2}\right) dx
\end{align*}
\end{frame}

\begin{frame}{Grafiki}
\begin{center}
\includegraphics[width=\textwidth]{./graf/politechnika_sl_logo_poziom_pl.eps}
\end{center}
\end{frame}

\section{Bloki}

\begin{frame}{Bloki}
\begin{definition}[Koercja]
Spiralna dyslokacja efemerydy Kuppelweisera
\end{definition}

\begin{exampleblock}{Przykład}
Przykładowo, gdy $x < \pi$, wtedy $\sum_{i=1}^{n} \frac{1}{i}$.
\end{exampleblock}

\begin{block}{Tytuł bloku}
Zawartość bloku …
\end{block}

\begin{alertblock}{Uwaga!}
Memento mori!
\end{alertblock}

\end{frame}

\begin{frame}[standout]
slajd negatywowy
\end{frame}

\maketitle

\appendix 

\begin{frame}{Dodatek}
\begin{enumerate}
\item Slajdy w dodatku nie są wliczane do liczby stron prezentacji. 
\item Nie mają numerów stron.
\item Nie wpływają na pasek postępu slajdów.
\end{enumerate}
\end{frame}

\end{document}

